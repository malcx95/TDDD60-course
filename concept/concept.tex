\documentclass[a4paper,12pt,titlepage]{article}
\usepackage[utf8]{inputenc}
\usepackage{textcomp}
\usepackage{geometry}
\usepackage{fancyhdr}
\pagestyle{fancy}
\lhead{Malcolm Vigren}
\rhead{D3.c}
\geometry{margin=3cm}
\usepackage{parskip} 
\begin{document}
\section*{Koncept för Snowslide Car Edition}

Snowslide Car Edition är en underhållningstjänst tänkt att ta plats i bilar.
Pia identifieras som den primära personan, och Tom som den sekundära.

\subsection*{Radikala koncept}

\subsubsection*{Koncept 1: Snowslide4kidz}
Snowslide4kidz möjliggör att barn kan ha kul i bilen samtidigt som föräldrarna
har kontroll över det innehåll och mängden barnen konsumerar.
Föräldrar skapar helt enkelt profiler för sina barn (eventuellt olika profiler för varje
barn), där exempelvis parametrar baserad på bland annat åldersbegränsning kan ställas in
för vilken media ska kunna tillåtas för vilka profiler, huruvida köp inuti
appar och i medie-affären ska vara tillåtna, samt en maximal visningstid för
filmer eller speltid för spel per dag. Barnprofilerna kan tilldelas till vissa
säten i bilen, och kan enkelt justeras via drag and drop via huvudpekskärmen i
framsätet, samtidigt som de inte kan justeras av någon som inte har tillgång
till pekskärmen där fram.

Pias mål som konceptet bidrar till:
\begin{itemize}
    \item Bekvämlighet
    \item När hon reser med barnen vill
        hon att de (åtminstone Tom) ska vara
        sysselsatta så att resan blir lugn.
    \item Vill att Tom ska göra annat än att
        bara titta på teve eller spela datorspel.
    \item Vill inte att Tom ska titta på saker
        på teve eller nätet, eller spela spel,
        som inte är lämpliga för en nioåring.
\end{itemize}

Toms mål som konceptet bidrar till:
\begin{itemize}
    \item Hålla sig sysselsatt (alltså inte ha
        tråkigt): Göra det han måste för
        föräldrarna och skolan, men bli klar så
        fort som möjligt. Kolla sina
        favoritfilmer och teveserier, och chatta
        eller söka information om dem på
        nätet. Hitta riktigt roliga, men gratis,
        spel på nätet, så att han kan få nya
        upplevelser utan att behöva be om
        pengar för att köpa saker. 
    \item Vill kolla på sina favoritserier när
        han vill. Eller får för mamma och
        pappa.
\end{itemize}

Målgrupper: Föräldrar med barn.

\subsection*{Värderingsmatris}
% TODO text

\subsection*{Gränssnittsflöde}
% TODO text

\end{document}
