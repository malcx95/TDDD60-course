\documentclass[a4paper,12pt,titlepage]{article}
\usepackage[utf8]{inputenc}
\usepackage{textcomp}
\usepackage{geometry}
\usepackage{fancyhdr}
\pagestyle{fancy}
\lhead{Malcolm Vigren}
\rhead{D3.c}
\geometry{margin=3cm}
\usepackage{parskip} 
\begin{document}
\pagenumbering{gobble}
\section*{Litteratursammanfattning}
\subsection*{Konceptfasens idéer och värderingar}

När övergripande avsikter har framtagits, ska man börja skapa idéer och
designkoncept. Ett designkoncept är ett kommunikationsverktyg som uttrycker kärnan i en
produkt utan tekniska detaljer. 

Designperspektivet bestämmer karaktären som produkt- eller tjänstekonceptet är tänkt
att ha, och ställer frågan om vilka övergripande egenskaper som ska tas fram.
Verktygsperspektivet lyfter fram praktiska aspekter av interaktionen,
medan medieperspektivet ser designobjektet som medium för kommunikation.
I aktörsperspektivet ses datorn som aktör som tar kommandon av människan, i
systemperspektivet ses människan och datorer som komponenter i ett funktionellt
system, och i maskinperspektivet ses människan som operatör över datorn.

Vilken position produkten/tjänsten ska anta i användarens uppmärksamhet 
är också något som ska funderas över gällande konceptdesign.
En dominerande position ges ofta till applikationer med mycket funktionalitet,
medan en flyktig position ges till applikationer som gör en enstaka eller ett
fåtal saker. En applikation ges position i bakgrunden ifall användaren inte
interagerar direkt med den.

För inspirationens skull kan man göra en genreanalys, där man analyserar vad
konkurrenter eller förebilder gör. För att sätta slumpen i systemet och hjälpa
när kreativiteten stannat kan man
använda sig av olika metoder för generering av idéer, som brainstorming, metod
635, funktionsdriven divergens, kvalitetsdriven divergens eller metafordriven
divergens. Dessa tekniker genererar radikala koncept, som inte bör direkt
avfärdas utan bör lekas med och se var de skulle passa.

När koncept har valts måste de jämföras och balanseras, för att få en referenspunkt som
styr utvecklingen i en bra rikting. Värderingsmatriser är ett verktyg för detta.
När designgruppen har identifierat en eller flera tydliga användningsfall kan storyboards
eller skrivna scenarion tas fram.

\subsection*{Bearbetningsfasen}

I bearbetningsfasen tas funktioner och innehåll fram baserat på koncepten genererade
under konceptfasen. I slutet ska designen göras till en första prototyp. 
Ett uppgiftsflöde kan tas fram för att bryta ner scenarierna framtagna under konceptfasen,
och beskriva användningen i större detalj. 

\end{document}
