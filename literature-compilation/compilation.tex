\documentclass[a4paper,12pt,titlepage]{article}
\usepackage[utf8]{inputenc}
\usepackage{textcomp}
\usepackage{geometry}
\usepackage{fancyhdr}
\pagestyle{fancy}
\lhead{Malcolm Vigren}
\rhead{D3.c}
\geometry{margin=2.4cm}
\usepackage{parskip} 
\begin{document}
\pagenumbering{gobble}
\section*{Litteratursammanfattning}
\subsection*{Konceptfasens idéer och värderingar}

När övergripande avsikter har framtagits, ska man börja skapa idéer och
designkoncept. Ett designkoncept är ett kommunikationsverktyg som uttrycker kärnan i en
produkt utan tekniska detaljer. 

Designperspektivet bestämmer karaktären som produkt- eller tjänstekonceptet är tänkt
att ha, och ställer frågan om vilka övergripande egenskaper som ska tas fram.
Verktygsperspektivet lyfter fram praktiska aspekter av interaktionen,
medan medieperspektivet ser designobjektet som medium för kommunikation.
I aktörsperspektivet ses datorn som aktör som tar kommandon av människan, i
systemperspektivet ses människan och datorer som komponenter i ett funktionellt
system, och i maskinperspektivet ses människan som operatör över datorn.

Vilken position produkten/tjänsten ska anta i användarens uppmärksamhet 
är också något som ska funderas över gällande konceptdesign.
En dominerande position ges ofta till applikationer med mycket funktionalitet,
medan en flyktig position ges till applikationer som gör en enstaka eller ett
fåtal saker. En applikation ges position i bakgrunden ifall användaren inte
interagerar direkt med den.

För inspirationens skull kan man göra en genreanalys, där man analyserar vad
konkurrenter eller förebilder gör. För att sätta slumpen i systemet och hjälpa
när kreativiteten stannat kan man
använda sig av olika metoder för generering av idéer, som brainstorming, metod
635, funktionsdriven divergens, kvalitetsdriven divergens eller metafordriven
divergens. Dessa tekniker genererar radikala koncept, som bör inte direkt
avfärdas utan lekas med och se var de skulle passa.

När koncept har valts måste de jämföras och balanseras, för att få en referenspunkt som
styr utvecklingen i en bra rikting. Värderingsmatriser är ett verktyg för detta.
När designgruppen har identifierat en eller flera tydliga användningsfall kan storyboards
eller skrivna scenarion tas fram.

\subsection*{Bearbetningsfasen}

I bearbetningsfasen tas funktioner och innehåll fram baserat på koncepten genererade
under konceptfasen. I slutet ska designen ha gjorts till en första prototyp.
Designgruppen kan alltså nu efter konceptfasen måla upp en tydligare bild av
hur produkten eller tjänsten ska upplevas. Sätt som detta kan uppnås på är
exempelvis via uppritning av målträd med nyckelord, eller om applikationen är
en webbsida eller är hiearkiskt ordnad, kan en webbkarta göras, som beskriver
navigationsstrukturen av programmet.

Om designgruppen vet vilka funktioner, vilket innehåll och vilken struktur applikationen
ska ha är det passande att börja med gränssnittet. Detta kan exempelvis baseras
på de scenarion gruppen tagit fram. Saker som bör tänkas igenom är
bland annat de mekanismer gränssnittet använder för att få saker gjorda, och
vilka sorters gränssnittskomponenter som ska användas för dessa. Exempelvis kan
enkla val i gränssnittet utföras av knappar, radioknappar eller kryssrutor.

Det finns också många principer att hålla sig till gällande gränssnittsdesign.
Man använder till exempel handlingsinviter för att göra programmet
självinstruerande och naturligt. Metaforer kan användas för att göra
upplevelsen mer bekant genom liknelse med t ex ett fysiskt kontor. Återkoppling
ger tydlighet hos handlingars konsekvenser, och "Feed-forward" används för att
guida användaren om vad som finns att komma.

För att testa validiteten hos sin design används ofta pappersprototyper, som
består av skärmkomponenter gjorda i papper och kartong. Dessa är bra då de inte
är begränsade av nuvarande hård- och mjukvara och är enkla att modifiera
tillsammans med intressenter. Denna metod testar huvudsakligen flödet,
begripligheten och det övergripande konceptet, där deltagarna är representativa
för målgruppen.

Tester med pappersprototyper inleds med frågor om vad deltagaren tror att det är för
produkt och beskriva vad alla komponenter i gränssnittet gör. Designbrister
syns här om deltagarens beskrivning inte matchar vad gruppen tänkt. Deltagaren
blir sedan ombedd att utföra uppgifter nedskrivna på papper utan instruktioner,
där nästkommande skärmar inte är synliga. Det som observeras hos deltagaren är
främst vilka vägar som tas och i vilka situationer svårigheter uppstår. Efter
testet ställs utvärderande frågor och anteckningar samlas på indexkort.

\subsection*{Detaljeringsfasen, överlämning och avslutning}

Resultaten av bearbetningsfasens design, tillsammans med en kompletterande datainsamling,
görs till specifika mätbara användar- och verksamhetskrav
under detaljeringsfasen. Den kompletterande datainsamlingen 
specificerar de abstrakta kvalitativa kraven från bearbetningsfasen, genom
undersökning av krav om mängd, frekvens och längd. Efter detta definieras
märbara användar- och verksamhetskrav.

Utseende och känsla är vad detaljeringsfasens idéarbete främst handlar om.
Tidigare har de visuella delarna varit skissartade, men nu ska beslut om vad
produkten ska likna fattas. Exempel på aspekter att tänka på är visuell ordning
av saker i grupperingar, borttagning av saker som inte behövs, placering av viktiga saker
centralt eller högt upp samt tydlig segmentering av skärmbilden. Olika
alternativ av interaktivitetsattributen \textit{samtidighet,
kontinuitet, förutsägbarhet, rörelse, hastighet, exakthet} och
\textit{responsivitet}, bör också övervägas.

Datorprototyper gör det möjligt att testa designens utseende och känsla. En så
bra prototyp som tidsbudgeten tillåter bör göras, dock inte mer än vad som
krävs för kommunikation, specifikation och testning av designen. En evolutionär
prototyp är en prototyp som är byggd i samma programmeringsmiljö som
slutprodukten. Denna typ av prototyp tar dock ofta längre tid att tillverka än
en som ska kasseras.

Designgruppen avgör om produkten förbättrats jämfört med tidigare i en
avslutande använd-barhetstestning, som liknar de tester som utfördes under
bearbetningsfasen. Även så kallade reaktionskort kan användas, där testanvändaren
väljer kort ur en kortlek med egenskaper som beskriver vad de tyckte om systemet.

När designen är godkänd av intressenterna överlämnas det för implementation. Designers
kan dock fortfarande behövas, för att lösa små grännsittsfrågor eller producera
komponenter och media. Varje designprojekt bör dessutom avslutas med en
reflektion om de lärdomar som gjorts.

\end{document}
