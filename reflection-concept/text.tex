Jag valde Designuppdrag 2: Snowslide Car Edition, i vilket ett
underhållningssystem för bilar skulle konstrueras. Personorna som
designuppdraget riktade sig mot var Pia och Tom. Pia är en förälder 
som ofta reser med jobbet och åker långa bilresor privat till släktingar. Tom
är Pias nioåriga son som gillar roliga prylar.

Det designkoncept som valdes var Snowslide4U. Detta koncept gör att användare
konsumerar sin media genom egna profiler, där de kan exempelvis skapa egna
spellistor i musikspelaren, bokmärken i webbläsaren eller ladda ner egna appar.
Konceptet ger också föräldrar till barn möjligheten att begränsa deras
mediakonsumtion, då de kan skapa barnkonton åt sina barn. Dessa konton kan
ställas in så att endast en viss tidsmängd per dag kan spenderas på att spela
spel eller titta på film, och endast filmer och appar med högst en viss
åldersgräns kan konsumeras.

\subsection*{Konceptgenerering}

Konceptgenereringen började genom ett brainstormingstillfälle med andra
studenter som går samma kurs, under vilket Metod 635 (Arvola 2014, 95) användes
för att generera idéer om koncept. Dessa idéer användes sedan för att generera
3 radikala koncept. Dessa var \textit{Snowslide4U}, ett koncept centrerat kring
användarkonton, \textit{Snowslide Connected}, fokuserande på bilens anslutning
till andra bilar och Internet, samt \textit{Snowslide Assistant}, där
röststyrning av underhållningen är en central funktion. Storyboards för varje
koncept skapades som illustration av tänkt användning.

Jag tyckte att det var svårt att komma på idéer på koncept och funktioner, både
under brainstormingstillfället och generering av radikala koncept. När jag hade
fått en grov idé om hur varje koncept skulle se ut blev det dock enklare att
bestämma vilka funktioner dessa skulle ha på konceptnivå.

Det som jag lyckades ganska bra med var förmodligen att jag kunde ganska bra
koppla de designmål som strävades efter till konceptens funktioner. Jag tror
också att jag lyckades ganska bra med att välja funktioner som är legitimt
användbara, och inte bara ``ser coola ut''. Det som jag skulle kunna ha
förbättrat är själva idégenereringen, är att ge mer ``galna'' och mindre
självklara förslag.



