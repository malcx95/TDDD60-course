\documentclass[a4paper,12pt,titlepage]{article}
\usepackage[utf8]{inputenc}
\usepackage{textcomp}
\usepackage{geometry}
\usepackage{fancyhdr}
\usepackage[yyyymmdd]{datetime}
\usepackage{tabularx}
\usepackage{parskip} 
\usepackage{longtable}
\usepackage{graphicx}
\usepackage{booktabs}
\pagestyle{fancy}
\lhead{Malcolm Vigren}
\rhead{D3.c}
\geometry{margin=3cm}
\renewcommand{\dateseparator}{--}

\begin{document}

\section*{Användarbarhetstest för Snowslide Car Edition}

\renewcommand*{\arraystretch}{1.4}
\begin{longtable}[l]{p{4cm} l}
    \textbf{Rapportdatum}    & \today \\
    \textbf{Testdatum   }    & 2017--03--01 \\
    \textbf{Utförd av   }    & Malcolm Vigren \\
    \textbf{Epost       }    & malvi108@student.liu.se \\
\end{longtable}

\subsection*{Sammanfattning}
\subsection*{Metod}
\subsubsection*{Vilka som testades}
2 personer testades. Personen som agerade som Pia var mycket representativ för
sin roll. Personen som spelade Tom var dock mindre representativ, då hon var en
kvinna i 20-års-åldern. Detta bedömdes dock vara acceptabelt, då Tom är
sekundär persona som är relativt datorvan, och en 90-talist borde ha ungefär
liknande nivå av datorvana.

\begin{longtable}[l]{c c c p{4cm} c}
    \textbf{Testperson} & \textbf{Persona} & \textbf{Ålder} &
    \centering \textbf{Datoranvändning (tim/vecka)} & \textbf{Kön} \\ \midrule
    1 & Pia & Över 40 & \centering 40--50 & Kvinna \\ \midrule
    2 & Tom & 18--25 & \centering 30--40 & Kvinna \\ \midrule
\end{longtable}

\subsection*{Större fynd och rekommendationer}
\subsection*{Detaljerade fynd och rekommendationer}

\end{document}
